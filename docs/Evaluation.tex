# Testing

To have proper software engineering design, I have written test cases to ensure the analysis is running correctly. However, the approach to test is somewhat different from normal program. Since the \texttt{JsTainter} heavily relies on \texttt{Jalangi2} framework, it's quite hard to simply import the taint analysis unit and perform unit testing only on that unit. The reason is that \texttt{Jalangi2} has done many things for \texttt{JsTainter}, and \texttt{JsTainter} does not work without this framework.

Therefore, instead, I have formulated a way to perform testing. Since we can instrument the JavaScript program that the analysis is running on, we can instrument on function call to perform our assertion.

\subsection{Checking Taint State}

\begin{minted}{javascript}
var a; //something whose taint state is to be examined
const assertTaint = "assertTaint";
assertTaint(a, true);
\end{minted}

The codes above will simply throw an exception in normal execution. However, if it is instrumented by our analysis program, we can examine the value of \texttt{function} parameter in the function call instrumentation callback. This value should be a \texttt{function} type variable in normal case, but if it is a \texttt{string} type variable and has value \texttt{"assertTaint"}, then we know we are going to perform assertion against the taint state of given variable, instead of executing the function call that will throw the error.

In \texttt{assertTaint} function, the main goal is to check if \texttt{shadow(val)} (actual shadow value) is same as \texttt{taint} (expected shadow value). If they are not exactly same, assertion will fail. Variable \texttt{position} is the position of instruction that will be printed if assertion fails, which makes debug more convenient.

\begin{minted}{javascript}
function myAssert(b, position)
{
    if (b !== true)
    {
        Log.log("Assertion failure at" + JSON.stringify(position));
        assert(false);
    }
}
function assertTaint(val, taint, position)
{
    taint = actual(taint);
    // taint might be wrapped by AnnotatedValue, just in case
    const s = shadow(val);
    myAssert(typeof s === typeof taint, position);
    // type must be identical
    if (Array.isArray(s))
    {// if shadow value is array, all elements must be same
        myAssert(s.length === taint.length, position);
        for (var i = 0; i < s.length; i++)
        {
            myAssert(s[i] === taint[i], position);
        }
    }
    else
    {// for any other cases such as basic-type case, shadow must be equal
        myAssert(s === taint, position);
    }
}

//in the instrumentation callback handler of function call
if (f === 'assertTaint')
{
    assertTaint(args[0], args[1], getPosition(iid));
}
\end{minted}

Note that these 2 pieces of codes above are in different files. The first code piece is in the JavaScript file that is going to be analyzed (e.i. \texttt{test.js}); while the second code piece is in the file that performs the dynamic taint analysis (e.i. \texttt{DynTaintAnalysis.js}) . Therefore, even if we have same \texttt{assertTaint} name as identifier in both files, there will not be any conflict.

\subsection{Checking Real Value}

Using the similar technique, real value of variable can also be checked. \texttt{"assert"} can be used to examine the correctness of real value of variable, and it is handled in the same way as \texttt{"assertTaint"}. 

\begin{minted}{javascript}
else if (f === "assert")
{
    myAssert(actual(args[0]), getPosition(iid));
    return {result : undefined};
}
\end{minted}

The usage of \texttt{assert} is a little bit different from \texttt{assertTaint}. Because real value can be directly access by program that is being analyzed, comparison can be done in the JavaScript program. For example,

\begin{minted}{javascript}
const assert = "assert";
assert(a == 1);
\end{minted}

\subsection{Evaluation on Basic Test Cases}

\subsubsection{Implementation}

To test the correctness of \texttt{JsTainter}, I have written many test cases in directory \texttt{tests/}. This will be tested by a simple Python script.

\end{minted}python
from os import system,walk
from re import search
cmd = "node jalangi2/src/js/commands/jalangi.js --inlineIID --inlineSource --analysis jalangi2/src/js/sample_analyses/ChainedAnalyses.js --analysis Utils.js --analysis Log.js --analysis TaintLogic.js --analysis NullBrowser.js --analysis DefaultConfig.js --analysis DynTaintAnalysis.js tests/%s"

i = 0
for root,subdirs,files in walk("./tests/"):
    i += 1
assert i == 1

for f in files:
    ret = search("^test[a-zA-Z0-9]+\\.js$", f)
    if ret: # iterate file with format testxxx.js
        print "Testing file: " + f
        ret = system(cmd % f) # execute analysis
        if ret != 0:
            print "Error in file %s" % f
            exit(-1)
\end{minted}

The reason why regular expression is used to filter file name is that Jalangi2 will generate some temporary files in that directory when analysis is performed, such as \texttt{testxxx_jalangi_.js} and \texttt{testxxx_jalangi_.json}, and only files with correct file name format should be analyzed and tested.

\subsubsection{Tests}

There are many test cases, and I will discuss them one by one.

\texttt{testarith.js} is used to test the taint propagation of arithmetic operation, especially when one of the operand is tainted string. For example, \texttt{(taintedStr + '123' + taintedStr) \textit{ 7} should be tainted because the result of this operation can be affected by \texttt{taintedStr} if it is a numeric string; while \texttt{(taintedStr + '0x' + taintedStr) } 7} should not be tainted because it always gives \texttt{NaN} no matter how \texttt{taintedStr} changes.

\texttt{testarithAdd.js} is used to test correctness of taint propagation of \texttt{add} operation.

\texttt{testbitoper.js} and \texttt{testshift.js} are used to test correctness of taint propagation of bit-wise operation, cases that operands are types other than number are also considered here.

\texttt{testcharAt.js}, \texttt{testindexOf.js} and \texttt{testsubstr.js} are used to test correctness of taint propagation of function \texttt{String.prototype.charAt}, \texttt{String.prototype.indexOf} and \texttt{String.prototype.substr} respectively, cases that arguments are types other than number are also considered here.

\texttt{testconstructor.js} is used to test taint propagation in JavaScript class. For example, when argument passed to constructor is tainted and is used to initialize the member fields, the fields should also be tainted. Also, \texttt{with} statement is also tested here.

\texttt{testeval.js} is used to test \texttt{eval} statement. In other word, taint propagation must also works well even if the statement that causes the taint propagation is executed using \texttt{eval}.

\texttt{testException.js} is used to test the case that when a tainted variable is thrown, the \texttt{catch} statement that receive the variable being thrown must also get the tainted value. //todo drawback

\texttt{testfield.js} is used to test correctness of taint propagation of putting field and getting field.

\texttt{testforinObject.js} is used to test correctness of \texttt{for in object} loop, which are properly handled by \texttt{analysis.forinObject} instrumentation callback function.

\texttt{testfunc.js} is used to test non-constructor function call, including anonymous function. //todo drawback

\texttt{testNumber.js} is used to test \texttt{Number} function. For example, when tainted string is casted to number by \texttt{Number} function, the return value should be tainted if it is controllable by the tainted argument.

\texttt{testConcat.js} is used to test string concatenation by operator \texttt{+}.

\section{Evaluation on Website}

In this section I am going to evaluate my analysis using web JavaScript program instead of \texttt{node.js} program. 

\subsection{Simple Example}

I have written a simple website that can be used to evaluate the effectiveness of taint analysis over JavaScript program with multiple sources and sinks, shown below.

\end{minted}html
<div id="sth"></div>
<script type="text/javascript">
function myclick()
{
    const url = window.location.href;
    const idx = url.indexOf('#')
    var hello, n;
    if (idx === -1)
    {
        hello = "";
    }
    else
    {
        n = url.substr(idx + 1);
        const desc = prompt("Please input the description: ");
        hello = "name:" + unescape(n) + " desc:" + desc;
    }
    const num1 = document.getElementById("text1").value;
    const num2 = document.getElementById("text2").value;
    const sum = Number(num1) + Number(num2);
    hello += " sum:"
    hello += sum
    document.getElementById("sth").innerHTML = hello;
    const req = new XMLHttpRequest();
    req.open("POST", window.location.origin + '/' + n);
    req.send(sum.toString())
}
</script>
<form>
    <input type="text" id="text1">
    <input type="text" id="text2">
    <input type="button" value="click me" onclick="myclick()">
</form>
\end{minted}

The result will be printed in \texttt{console} dialog, which is a list of JSON strings. In the following section, I will try to explain the result of the taint analysis and compare the result with real behavior of the JavaScript program, so that effectiveness of taint analysis can be evaluated.

Firstly, URL is fetched, which is partially tainted, and written to variable \texttt{url}. The \texttt{id} of this source is 0, so \texttt{taint information variable} for tainted character is \texttt{1 << 0 == 1} (\texttt{number} type is used to implement boolean array).  This behavior is recorded properly.

\end{minted}json
{"type":"source","typeOf":"string","file":"91ea3bc2825abb590247bbfa10e8631f.js","pos":[4,17,4,37],"name":"href","id":0}
{"type":"write","typeOf":"string","file":"91ea3bc2825abb590247bbfa10e8631f.js","pos":[4,17,4,37],"name":"url"}
\end{minted}

The file name is a MD5 hash, which is generated by Jalangi2, and is indeed a bit weird. However, we only need to know that the file represents the JavaScript code in the \texttt{<script>} tag shown above. The value in this property is same for all JSON results, so to make the report more clear, I will delete this field in following section, but actually they still exist.

Then, \texttt{indexOf} is called on \texttt{url} and the return value is tainted, which is then written to variable \texttt{idx}. This is intended result because \texttt{idx} can vary as the user input changes. For example, since there could be a query string before the first \texttt{#}, and its length is dependent on user, which means the value of \texttt{idx} can be controlled by user and thus should be tainted.

\end{minted}json
{"type":"read","typeOf":"string","pos":[5,17,5,20],"name":"url"}
{"type":"write","typeOf":"number","pos":[5,17,5,33],"name":"idx"}
\end{minted}

Then, variable \texttt{idx} is read and used in a \texttt{if} condition, and corresponding \texttt{log} message is produced, which again is the intended result.

\end{minted}json
{"type":"read","typeOf":"number","pos":[7,9,7,12],"name":"idx"}
{"type":"log","pos":[7,9,7,19],"msg":"Tainted variable false being used in conditional"}
\end{minted}

Then the program goes into the \texttt{else} branch. Function \texttt{substr} function is called using \texttt{url} and \texttt{idx}, and its return value, which is tainted, is assigned to variable \texttt{n}. The corresponding results are still correct.

\end{minted}json
{"type":"read","typeOf":"string","pos":[13,13,13,16],"name":"url"}
{"type":"read","typeOf":"number","pos":[13,24,13,27],"name":"idx"}
{"type":"write","typeOf":"string","pos":[13,13,13,32],"name":"n"}
\end{minted}

At line 14, input is obtained from \texttt{prompt} function again and assigned to variable \texttt{desc}, and at line 15, tainted variable \texttt{n} and \texttt{desc} are used to generate a partially tainted string, which is assigned to variable \texttt{hello}. The \texttt{id} of this source is 1, so \texttt{taint information variable} for every character is \texttt{1 << 1 == 2}. These behaviors are all recorded properly. 

\end{minted}json
{"type":"source","typeOf":"string","pos":[14,22,14,62],"name":"prompt","id":1}
{"type":"write","typeOf":"string","pos":[14,22,14,62],"name":"desc"}
{"type":"read","typeOf":"string","pos":[15,36,15,37],"name":"n"}
{"type":"read","typeOf":"string","pos":[15,52,15,56],"name":"desc"}
{"type":"write","typeOf":"string","pos":[15,17,15,56],"name":"hello"}
\end{minted}

After the \texttt{else} block, input string is obtained from 2 \texttt{input} tags, and assigned to variable \texttt{num1} and variable \texttt{num2}. There are also \texttt{log} type JSONs that has been recorded, since \texttt{getElementById} native function is not handled in \texttt{invokeFun} handler; but actually this function does not need to be handled, so this JSON can be ignored. In addition, even if the inputs are both from \texttt{<input>} field, the \texttt{id} numbers being allocated are different, thanks to the \texttt{id} allocator. 

\end{minted}json
{"type":"log","pos":[17,18,17,50],"msg":"Unhandled native function getElementById"}
{"type":"source","typeOf":"string","pos":[17,18,17,56],"name":"value","id":2}
{"type":"write","typeOf":"string","pos":[17,18,17,56],"name":"num1"}
{"type":"log","pos":[18,18,18,50],"msg":"Unhandled native function getElementById"}
{"type":"source","typeOf":"string","pos":[18,18,18,56],"name":"value","id":3}
{"type":"write","typeOf":"string","pos":[18,18,18,56],"name":"num2"}
\end{minted}

The \texttt{num1} and \texttt{num2} are converted to number, and used to calculate \texttt{sum}, which is tainted and is concatenated to string \texttt{hello} in next step. Note that \texttt{hello += sth} is identical to \texttt{hello = hello + sth}, so a \texttt{read} record on \texttt{hello} variable will also be recorded, which is not a mistake.

 \end{minted}json
{"type":"read","typeOf":"string","pos":[19,24,19,28],"name":"num1"}
{"type":"read","typeOf":"string","pos":[19,39,19,43],"name":"num2"}
{"type":"write","typeOf":"number","pos":[19,17,19,44],"name":"sum"}
{"type":"read","typeOf":"string","pos":[20,5,20,10],"name":"hello"}
{"type":"write","typeOf":"string","pos":[20,5,20,21],"name":"hello"}
{"type":"read","typeOf":"string","pos":[21,5,21,10],"name":"hello"}
{"type":"read","typeOf":"number","pos":[21,14,21,17],"name":"sum"}
{"type":"write","typeOf":"string","pos":[21,5,21,17],"name":"hello"}
 \end{minted}

Here is the first sink being detected: variable \texttt{hello} is written to \texttt{innerHTML} field of a \texttt{<div>} DOM object. The value being written to the sink and corresponding shadow value are presented in JSON.

The 4 \texttt{1}s correspond to \texttt{"2019"}, which comes from source with \texttt{id==0}, URL; the 3 \texttt{2}s correspond to \texttt{"AAA"}, which comes from source with \texttt{id==1}, return value of \texttt{prompt}. The 2 \texttt{12}s are interesting: since \texttt{24} results from adding values from 2 input tags, it can be affected by both source with \texttt{id==2} and source with \texttt{id==3}, and \texttt{12} is the result from \texttt{(1<<2) | (1<<3) == 4 | 8 == 12}.

\end{minted}json
{"type":"log","pos":[22,5,22,35],"msg":"Unhandled native function getElementById"}
{"type":"read","typeOf":"string","pos":[22,48,22,53],"name":"hello"}
{"type":"sink","pos":[22,5,22,53],"value":"name:2019 desc:AAA sum:24","shadow":[0,0,0,0,0,1,1,1,1,0,0,0,0,0,0,2,2,2,0,0,0,0,0,12,12],"name":"[object HTMLDivElement].innerHTML"}
\end{minted}

Then, to test native function sink, \texttt{XMLHttpRequest} is used. The result is also correct.

\end{minted}json 
{"type":"log","pos":[23,17,23,37],"msg":"Unhandled native function XMLHttpRequest"}
{"type":"read","typeOf":"string","pos":[24,53,24,54],"name":"n"}
{"type":"sink","pos":[24,5,24,55],"value":["POST","https://www.doc.ic.ac.uk/2019"],"shadow":[[0,0,0,0],[0,0,0,0,0,0,0,0,0,0,0,0,0,0,0,0,0,0,0,0,0,0,0,0,0,1,1,1,1]],"name":"open"}
{"type":"read","typeOf":"number","pos":[25,14,25,17],"name":"sum"}
{"type":"sink","pos":[25,5,25,29],"value":["24"],"shadow":[[12,12]],"name":"send"}
\end{minted}

\section{Weakness}

\subsection{Implicit Flow}

To detect implicit information flow, JavaScript must be analyzed from high-level perspective. However, since I have applied pure dynamic analysis, analysis can only be performed for each individual JavaScript operation. Therefore, automatic detection of implicit flow is not possible.

\subsection{Unable to Track Taint of Native Object Fields}

In JavaScript, there are some native objects. For example, \texttt{Error} is the object that is used to throw an exception, and can be used in this way:

\begin{minted}{javascript}
try
{
    throw new Error("some message");
}
catch (e)
{
    console.log(e.message);
}
\end{minted}

There might be cases that the string passed into \texttt{Error} is tainted. However, unlike non-native classes, Jalangi2 cannot instrument into constructor of \texttt{Error}, thus unable to tackle the taint state of \texttt{message} field. Thus, false negative would be produced.

\subsection{Unable to Execute Codes with Lambda Function}

This is actually a problem from Jalangi2 instead of JsTainter. In Jalangi2, lambda expression is wrongly instrumented: the parameters of lambda expression are instrumented like variable, which causes JavaScript grammar to be wrong. For [example](https://github.com/Samsung/jalangi2/issues/149), when expression \texttt{const lambda = (a,b) => a+b;} is instrumented, \texttt{(a,b)} part would be instrumented to \texttt{(J$.R(81, 'a', 'a', 1), J$.R(89, 'b', 'b', 1))}, which is certainly wrong because this is not variable read and should not be modified, just like \texttt{(a,b)} part in \texttt{function (a,b) {return a+b}}.

\subsection{Detectable by Program being Analyzed}

For some JavaScript programs, anti-debug techniques are applied to prevent people from reverse engineering the product. For example, JavaScript program can convert function to string and check if the function is modified.

\begin{minted}{javascript}
function some_func() { /\textit{code that does not want to be modified by reverse engineer}/ }
const correct_crc = 0x708D2F22; /\textit{ crc32 value of String(some_func) }/
function crc32(str) { /\textit{code that implements crc32 hash algorithm}/ }
if (crc32(String(some_func)) != correct_crc)
    throw Error("Hack3r detected!")
\end{minted}

If \texttt{some_func} function is instrumented by Jalangi2, the CRC-32 value will a become different one, so an exception will be thrown, which means the behavior of the program becomes different after instrumentation. This is not desirable.

\subsection{Prototype Poisoning}

In current implementation, prototype poisoning is not properly handled. For example, behavior of field setting can be modified to particular function by using 

\begin{minted}{javascript}
Object.defineProperty(SomeClass.prototype, "key", {set:function(){console.log("1337")}})
\end{minted}

After this statement is executed, if \texttt{obj} is an instance of \texttt{SomeClass}, and \texttt{obj["key"]=1} is executed, instead of executing normal field setting, \texttt{function(){console.log("1337")}} will be executed, so \texttt{obj["key"]} will still be \texttt{undefined}. In this case, tracking the shadow value of the object in the normal way might cause inaccuracy.



1. eval on self written web site
2. eval on web CTF challenge?
3. eval on real world website
4. eval on usability, e.g. environment congifuation
5. drawback: anti-instrumentation
6. 


